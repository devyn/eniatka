%% LyX 1.6.4 created this file.  For more info, see http://www.lyx.org/.
%% Do not edit unless you really know what you are doing.
\documentclass[letterpaper,english]{article}
\usepackage[T1]{fontenc}
\usepackage[latin9]{inputenc}

\makeatletter
%%%%%%%%%%%%%%%%%%%%%%%%%%%%%% User specified LaTeX commands.


\title{eniatka}
\author{Devyn Cairns}

\makeatother

\usepackage{babel}

\begin{document}

\title{eniatka}


\author{Devyn (yas'nawku-kodto-linga) Cairns}

\maketitle
To build a syntactically unambiguous language, like Lojban. Unlike
Lojban however, I am not trying to build an auxilary language.


\section{alphabet}

Alphabet in eniatka is binary. Letters are either of level 0 or level
1 (two groups, vowels + consonants).\\


0: i e o u a

1: k s r y t w d g n l x p\\


That's a size 12 alphabet in total. Two level 0 letters can be joined
to form one level 0 letter, sounding like a dipthong. Double level
zeroes means you hold for twice normal length (what you personally
would usually hold for). Double level ones are not allowed.


\section{word format}

All words in eniatka follow strict formatting rules. This is to enforce
unambiguity. These formats are:\\


010110, 10110, 11010, 1010, 010, 101\\


Examples (a dot designates a dipthong):\\


eniatka => 010.110

sulxao => 10110.

nwautai => 110.10.

rewae => 1010.

otoa => 010.

let => 101\\


This way, there is no confusion over what is a word and what isn't.

For words (e.g. people, places, other names) from other languages:
There is no restriction on format, but you must put colons before
and after the word. Example: :dexen keraens:, a transliteration of
my name.


\section{complex word forming}

Obviously you will want to describe words. To just say {}``plane''
doesn't specify whose plane it is, nor what color it is.\\


nista => plane

nista-raulsi => white plane

sin'nista-raulsi => that white plane

aku'nista-tsauri-seogna => his nice blue plane\\


Breaking this down, prefixes are belongs\_to statements, and suffixes
are columns.


\section{writing a sentence}

Let's try something a bit more complex.\\


lan sin'nista yaka. ipi-raulsi ban. => I that plane admire. Blue It
is.


\section{names}

First, you should know what the meaning of your name is. If you don't,
either search it on the internet, or make one up for your self.

Names are generally constructed by taking a word for the meaning,
and appending suffixes for extra meaning, like titles. Using a title
with someone's name isn't required, but it can be used to clarify
exactly who or what you're talking about.\\


For example, I know that my name {}``Devyn'' translates to something
along the lines of {}``all-teacher''. So I translate that: yas'nawku
(teacher of all)

Next, I would take my hobbies, programming and linguistics, and append
them as suffixes: -kodto -linga

Finally, we get yas'nawku-kodto-linga. You could just say {}``yas'nawku-kodto''
or {}``yas'nawku-linga'' or even plain {}``yas'nawku'', but it
could be confusing in a large room of people. Suffixes are optional,
though.\\


So if I wanted to say what my name is, I could say {}``yas'nawku-kodto-linga
asra'ban''


\section{grammar ideas}

These are just basically my ideas for grammar.

A..arg T..arg-append X..fun\\


eniatka seogna => A X B C D E...

eniatka seogna, yisa. => A (X, Y...)

eniatka seogna lan, yisa lan => A (X B C D E..., Y B C D E...)

eniatka seogna, yisa; lan => A (X B C D E..., Y B C D E...) ; T U
V W...

osa'lan silpai apiogus, pol'kanat; tal'saila. => Cat{[}\{of\}me, (plays\{with\}
\& \{after\}eats), \{of-its\}food{]} :: My cat plays with and then
eats its food.

sin'asaipsu altain ikaesne-tap'isua, nikka aye'tokku. ipi kana plesa.
=> \{that\}Human{[}moves \{to\}grocery store\{the favorite\}, browse
everything{]} it{[}purchases food{]}

yas'nawku-kodto-linga asra'ban => All-teacher (of code and language)
self-am. => I am Devyn (of code and language).

I think I like this one...
\end{document}
